\documentclass[11pt, letterpaper, oneside]{memoir}

\usepackage[T1]{fontenc}
\usepackage{mathpazo}

% To create lists
\usepackage{enumitem}

% For header and footer
\makepagestyle{plain}
\makeevenfoot{plain}{}{\footnotesize Based on the book Calculus Volume 1. Download for free at https://openstax.org/details/books/calculus-volume-1.}{}
\makeoddfoot{plain}{}{\footnotesize Based on the book Calculus Volume 1. Download for free at https://openstax.org/details/books/calculus-volume-1.}{}

\pagestyle{plain}

\setlrmarginsandblock{3.5cm}{3.5cm}{*}
\setulmarginsandblock{3.5cm}{3.5cm}{*}
\setheadfoot{24pt}{48pt}
\checkandfixthelayout

\begin{document}

\chapter{Functions and Graphs}

\section*{Checkpoint Solutions}

\subsection*{Checkpoint 1.1: Evaluating Functions}

\subsubsection*{Instruction}

For the function $ f(x) = x^2 - 3x + 5 $ evaluate

\begin{enumerate}[label=(\alph*)]
  \item $ f(1) $
  \item $ f(a + h) $
\end{enumerate}

\subsubsection*{Solution}

\begin{enumerate}[label=(\alph*)]
  \item $ f(1) = 1 ^ 2 - 3 \cdot 1 + 5 = 1 - 3 + 5 = 3 $
  \item $ f(a + h) = (a + h)^2 - 3(a + h) + 5 = a^2 + 2ah + h^2 - 3a - 3h + 5 $
\end{enumerate}

\subsubsection*{Answer}

\begin{enumerate}[label=(\alph*)]
  \item $ f(1) = 3 $
  \item $ f(a + h) = a^2 + 2ah + h^2 - 3a - 3h + 5 $
\end{enumerate}

\subsection*{Checkpoint 1.2: Finding Domain and Range}

\subsubsection*{Instruction}

Find the domain and range for $ f(x) = \sqrt{4 - 2x} + 5 $.

\subsubsection*{Solution}

\begin{enumerate}[label=\roman*]
  \item To find the domain of $ f $, we need the expression $ 4 - 2x \ge 0 $, due to that real negative numbers do not have a square root. Solving this inequality, we conclude that the domain is $ \{ x \mid x \le 2 \} $.
  \item To find the range of $ f $, we note that since $ \sqrt{4 - 2x} \ge 0 $, it follows that $ f(x) = \sqrt{4 - 2x} + 5 \ge 5 $. Therefore, the range of $ f $ must be a subset of the set $ \{ y \mid y \ge 5 \} $.

    To show that every element in this set is in the range of $ f $, we need to show that for all $ y $ in this set, there exists a real number $ x $ in the domain such that $ f(x) = y $. Let $ y \ge 5 $. Then, $ f(x) = y $ if and only if
    $$ \phantom{.}
    \sqrt{4 - 2x} + 5 = y
    .$$
    Solving this equation for $ x $, we see that $ x $ must solve the equation
    $$ \phantom{.}
    \sqrt{4 - 2x}= y - 5
    .$$
    Since $ y \ge 5 $, such an $ x $ could exist. Squaring both sides of the above equation we have
    $$ \phantom{.}
    4-2x = (y - 5)^2
    .$$
    Therefore we need
    $$ \phantom{,}
    -2x= (y - 5)^2 - 4
    ,$$
    which implies
    $$ \phantom{,}
    x= 2 -\frac{(y - 5)^2}{2}
    .$$
    We just need to verify that $ x $ is in the domain of $ f $. Since the domain of $ f $ consists of all real numbers less or equal to $ 2 $, and
    $$ \phantom{,}
    2 -\frac{(y - 5)^2}{2} \le 2
    ,$$
    there does exist an $ x $ in the domain of $ f $. We conclude that the range of $ f $ is $ \{ y \mid y \ge 5 \} $.
\end{enumerate}

\subsubsection*{Answer}

Domain $ = \{ x \mid x \le 2 \} $, range $ = \{ y \mid y \ge 5 \} $

\subsection*{Checkpoint 1.3: Finding Zeroes}

\subsubsection*{Instruction}

Find the zeroes of $ f(x) = x^3 - 5x^2 + 6x $.

\subsubsection*{Solution}

The zeroes of a function are the values of $ x $ where $ f(x) = 0 $. To find the zeroes, we need to solve
$$ \phantom{.}
f(x) = x^3 - 5x^2 + 6x = 0
.$$
Factor out $ x $
$$ \phantom{.}
f(x) = x(x^2 - 5x + 6) = 0
.$$
We can continue factoring by pure inspection, with the goal of finding a pair of numbers that add up to $ -5 $ and whose product is $ 6 $. This pair of numbers turns out to be $ -2 $ and $ -3 $, leading to the factoring
$$ \phantom{.}
f(x) = x(x - 2)(x - 3) = 0
.$$
From the above complete factoring of $ f $, we conclude that there are three zeroes when $ x $ is 0, 2, and 3.

\subsubsection{Answer}

$ x = 0, 2, 3 $

\subsection*{Checkpoint 1.4: Combining Functions Using Mathematical Operations}

\subsubsection{Instruction}

For $ f(x) = x^2 + 3 $ and $ g(x) = 2x - 5 $, find $ (f/g)(x) $ and state its domain.

\subsubsection{Solution}

To find $ (f/g)(x) $ we write the function with the quotient operator
$$ \phantom{.}
\frac{f}{g}(x) = \frac{x^2 + 3}{2x - 5}.
$$
The domain of this function is $ \{ x \mid x \ne \frac{5}{2} \} $.

\subsubsection{Answer}

$ \frac{f}{g}(x) = \frac{x^2 + 3}{2x - 5} $. The domain is $ \{ x \mid x \ne \frac{5}{2} \} $.

\subsection*{Checkpoint 1.5: Compositions of Functions}

\subsubsection{Instruction}

Let $ f(x) = 2 - 5x $. Let $ g(x) = \sqrt{x} $. Find $ (f \circ g)(x) $.

\subsubsection{Solution}

$ (f \circ g)(x) = f(g(x)) = f(\sqrt{x}) = 2 - 5\sqrt{x} $

\subsubsection{Answer}

$ (f \circ g)(x) = 2 - 5\sqrt{x} $

\subsection*{Checkpoint 1.6: Application Involving a Composite Function}

\subsubsection{Instruction}

If items are on sale for 10\% off their original price, and a customer has a coupon for an additional 30\% off, what will be the final price for an item that is originally x dollars, after applying the coupon to the sale price?

\subsubsection{Solution}

Since the sale price 10\% off the original price, if an item is $ x $ dollars, its sale price is given by
$$ \phantom{.}
f(x) = 0.90x
.$$
Since the coupon entitles an individual to 30\% off the price of any item, if an item is $ y $ dollars, the price after applying the coupon, is given by
$$ \phantom{.}
g(y) = 0.70y
.$$
Therefore, if the price is originally $ x $ dollars, its price after applying the coupon to the sale price will be
$$ \phantom{.}
(g \circ f)(x) = g(f(x)) = (0.70)0.90x = 0.63x.
.$$

\subsubsection{Answer}

$ (g \circ f)(x) = 0.63x $

\section*{Exercise Solutions}

For the following exercises, (a) determine the domain and the range of each relation, and (b) state whether the relation is a function.

\subsection*{Exercise 1.1.1}

\subsubsection{Instruction}

Relation:

\begin{table}[h]
  \begin{tabular}{ r r }
    \hline
    $ x $ & $ y $ \\
    \hline
    $ -3 $ & $ 9 $ \\
    $ -2 $ & $ 4 $ \\
    $ -1 $ & $ 1 $ \\
    $ 0 $ & $ 0 $ \\
    $ 1 $ & $ 1 $ \\
    $ 2 $ & $ 4 $ \\
    $ 3 $ & $ 9 $ \\
    \hline
  \end{tabular}
\end{table}

\subsubsection{Solution}

\begin{enumerate}[label=(\alph*)]
  \item The domain of the relation is the set of unique $ x $ values,
    $$ \phantom{.}
    \{ -3, -2, -1, 0, 1, 2, 3 \}
    .$$
    The range of the relation is the set of unique $ y $ values,
    $$ \phantom{.}
    \{ 0, 1, 4, 9 \}
    .$$
  \item This relation is a function, each input is a assigned to exactly one output.
\end{enumerate}

\subsubsection{Answer}

\begin{enumerate}[label=(\alph*)]
  \item Domain = $ \{ -3, -2, -1, 0, 1, 2, 3 \} $, range = $ \{ 0, 1, 4, 9 \} $
  \item Yes, a function
\end{enumerate}

\subsection*{Exercise 1.1.2}

\subsubsection{Instruction}

Relation:

\begin{table}[h]
  \begin{tabular}{ r r }
    \hline
    $ x $ & $ y $ \\
    \hline
    $ -3 $ & $ -2 $ \\
    $ -2 $ & $ -8 $ \\
    $ -1 $ & $ -1 $ \\
    $ 0 $ & $ 0 $ \\
    $ 1 $ & $ 1 $ \\
    $ 2 $ & $ 8 $ \\
    $ 3 $ & $ -2 $ \\
    \hline
  \end{tabular}
\end{table}

\subsubsection{Solution}

\begin{enumerate}[label=(\alph*)]
  \item The domain of the relation is the set of unique $ x $ values,
    $$ \phantom{.}
    \{ -3, -2, -1, 0, 1, 2, 3 \}
    .$$
    The range of the relation is the set of unique $ y $ values,
    $$ \phantom{.}
    \{ -8, -2, -1, 0, 1, 8\}
    .$$
  \item This relation is a function, each input is a assigned to exactly one output.
\end{enumerate}

\subsubsection{Answer}

\begin{enumerate}[label=(\alph*)]
  \item Domain = $ \{ -3, -2, -1, 0, 1, 2, 3 \} $, range = $ \{ -8, -2, -1, 0, 1, 8\} $
  \item Yes, a function
\end{enumerate}

\subsection*{Exercise 1.1.3}

\subsubsection{Instruction}

Relation:

\begin{table}[h]
  \begin{tabular}{ r r }
    \hline
    $ x $ & $ y $ \\
    \hline
    $ 1 $ & $ -3 $ \\
    $ 2 $ & $ -2 $ \\
    $ 3 $ & $ -1 $ \\
    $ 0 $ & $ 0 $ \\
    $ 1 $ & $ 1 $ \\
    $ 2 $ & $ 2 $ \\
    $ 3 $ & $ 3 $ \\
    \hline
  \end{tabular}
\end{table}

\subsubsection{Solution}

\begin{enumerate}[label=(\alph*)]
  \item The domain of the relation is the set of unique $ x $ values,
    $$ \phantom{.}
    \{ 0, 1, 2, 3 \}
    .$$
    The range of the relation is the set of unique $ y $ values,
    $$ \phantom{.}
    \{ -3, -2, -1, 0, 1, 2, 3 \}
    .$$
  \item This relation is not function, due to that each input is not assigned to exactly one output. Take for example $ x = 1 $ that can cause both $ y = -3 $ and $ y = 1 $.
\end{enumerate}

\subsubsection{Answer}

\begin{enumerate}[label=(\alph*)]
  \item Domain = $ \{ 0, 1, 2, 3 \} $, range = \{ -3, -2, -1, 0, 1, 2, 3 \}
  \item No, not a function
\end{enumerate}

\end{document}
