\documentclass[12pt, letterpaper, oneside]{memoir}

% Related to math
\usepackage{amsmath,amssymb,amsfonts,amsthm}

% To create lists
\usepackage{enumitem}

% For header and footer
\makepagestyle{plain}
\makeevenfoot{plain}{}{\footnotesize Based on the book Calculus Volume 1. Download for free at https://openstax.org/details/books/calculus-volume-1.}{}
\makeoddfoot{plain}{}{\footnotesize Based on the book Calculus Volume 1. Download for free at https://openstax.org/details/books/calculus-volume-1.}{}

\pagestyle{plain}

\setlrmarginsandblock{3.5cm}{3.5cm}{*}
\setulmarginsandblock{3.5cm}{3.5cm}{*}
\setheadfoot{24pt}{48pt}
\checkandfixthelayout

\begin{document}

\chapter{Functions and Graphs}

\section*{Checkpoint Solutions}

\subsection*{1.1 Evaluating Functions}

\subsubsection*{Instruction}

For the function $ f(x) = x^2 - 3x + 5 $ evaluate

\begin{enumerate}[label=(\alph*)]
  \item $ f(1) $
  \item $ f(a + h) $
\end{enumerate}

\subsubsection*{Solution}

\begin{enumerate}[label=(\alph*)]
  \item $ f(1) = 1 ^ 2 - 3 \cdot 1 + 5 = 1 - 3 + 5 = 3 $
  \item $ f(a + h) = (a + h)^2 - 3(a + h) + 5 = a^2 + 2ah + h^2 - 3a - 3h + 5 $
\end{enumerate}

\subsubsection*{Answer}

\begin{enumerate}[label=(\alph*)]
  \item $ f(1) = 3 $
  \item $ f(a + h) = a^2 + 2ah + h^2 - 3a - 3h + 5 $
\end{enumerate}

\subsection*{1.2 Finding Domain and Range}

\subsubsection*{Instruction}

Find the domain and range for $ f(x) = \sqrt{4 - 2x} + 5 $.

\subsubsection*{Solution}

\begin{enumerate}[label=\roman*]
  \item To find the domain of $ f $, we need the expression $ 4 - 2x \ge 0 $, due to that real negative numbers do not have a square root. Solving this inequality, we conclude that the domain is $ \{ x \mid x \le 2 \} $.
  \item To find the range of $ f $, we note that since $ \sqrt{4 - 2x} \ge 0 $, it follows that $ f(x) = \sqrt{4 - 2x} + 5 \ge 5 $. Therefore, the range of $ f $ must be a subset of the set $ \{ y \mid y \ge 5 \} $.

    To show that every element in this set is in the range of $ f $, we need to show that for all $ y $ in this set, there exists a real number $ x $ in the domain such that $ f(x) = y $. Let $ y \ge 5 $. Then, $ f(x) = y $ if and only if
    $$ \phantom{.}
    \sqrt{4 - 2x} + 5 = y
    .$$
    Solving this equation for $ x $, we see that $ x $ must solve the equation
    $$ \phantom{.}
    \sqrt{4 - 2x}= y - 5
    .$$
    Since $ y \ge 5 $, such an $ x $ could exist. Squaring both sides of the above equation we have
    $$ \phantom{.}
    4-2x = (y - 5)^2
    .$$
    Therefore we need
    $$ \phantom{,}
    -2x= (y - 5)^2 - 4
    ,$$
    which implies
    $$ \phantom{,}
    x= 2 -\frac{(y - 5)^2}{2}
    .$$
    We just need to verify that $ x $ is in the domain of $ f $. Since the domain of $ f $ consists of all real numbers less or equal to $ 2 $, and
    $$ \phantom{,}
    2 -\frac{(y - 5)^2}{2} \le 2
    ,$$
    there does exist an $ x $ in the domain of $ f $. We conclude that the range of $ f $ is $ \{ y \mid y \ge 5 \} $.
\end{enumerate}

\subsubsection*{Answer}

Domain $ = \{ x \mid x \le 2 \} $, range $ = \{ y \mid y \ge -4 \} $

\subsection*{1.3 Finding Zeroes}

\subsubsection*{Instruction}

Find the zeroes of $ f(x) = x^3 - 5x^2 + 6x $.

\subsubsection*{Solution}

The zeroes of a function are the values of $ x $ where $ f(x) = 0 $. To find the zeroes, we need to solve
$$ \phantom{.}
f(x) = x^3 - 5x^2 + 6x = 0
.$$
Factor out $ x $
$$ \phantom{.}
f(x) = x(x^2 - 5x + 6) = 0
.$$
We can continue factoring by pure inspection, with the goal of finding a pair of numbers that add up to $ -5 $ and whose product is $ 6 $. This pair of numbers turns out to be $ -2 $ and $ - 3 $, leading to the factoring
$$ \phantom{.}
f(x) = x(x - 2)(x - 3) = 0
.$$
From the above complete factoring of $ f $, we conclude that there are three zeroes when $ x $ is 0, 2, and 3.

\subsubsection{Answer}

$ x = 0, 2, 3 $

\end{document}
