\documentclass{article}
% General document formatting
\usepackage[margin=0.7in]{geometry}
\usepackage[parfill]{parskip}
\usepackage[utf8]{inputenc}

% Related to math
\usepackage{amsmath,amssymb,amsfonts,amsthm}

% To create lists
\usepackage{enumitem}

% For header and footer
\usepackage{fancyhdr}

\newcommand*\footerfont{\footnotesize}

\fancyhf{}
\fancyfoot[C]{\footerfont Based on the book Calculus Volume 1, download for free at https://openstax.org/details/books/calculus-volume-1.}
\renewcommand\headrulewidth{0pt}
\pagestyle{fancy}

\begin{document}

\section*{Chapter 1: Functions and Graphs}

\subsection*{Checkpoint Solutions}

\subsubsection*{1.1 Evaluating Functions}

For the function $ f(x) = x^2 - 3x + 5 $ evaluate

\begin{enumerate}[label=(\alph*)]
  \item $ f(1) $
  \item $ f(a + h) $
\end{enumerate}

\subsubsection*{Solution}

\begin{enumerate}[label=(\alph*)]
  \item $ f(1) = 1 ^ 2 - 3 \cdot 1 + 5 = 1 - 3 + 5 = 3 $
  \item $ f(a + h) = (a + h)^2 - 3(a + h) + 5 = a^2 + 2ah + h^2 - 3a - 3h + 5 $
\end{enumerate}

\subsubsection*{1.2 Finding Domain and Range}

Find the domain and range for $ f(x) = \sqrt{4 - 2x} + 5 $.

\begin{enumerate}[label=\roman*]
  \item To find the domain of $ f $, we need the expression $ 4 - 2x \ge 0 $, due to that real negative numbers do not have a square root. Solving this inequality, we conclude that the domain is $ \{x \mid x \le 2\} $.
  \item To find the range of $ f $, we note that since $ \sqrt{4 - 2x} \ge 0 $, $ f(x) = \sqrt{4 - 2x} + 5 \ge 5 $. Therefore, the range of $ f $ must be a subset of the set $ \{ y \mid y \ge 5\} $.
\end{enumerate}

\end{document}
