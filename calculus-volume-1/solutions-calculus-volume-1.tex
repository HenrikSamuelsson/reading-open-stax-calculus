\documentclass[11pt, letterpaper, oneside]{memoir}

\usepackage{../meta/calculus-volume-x}

\begin{document}

\chapter{Functions and Graphs}

\section{Review of Functions}

\subsection*{Checkpoint 1.1: Evaluating Functions}

\subsubsection*{Instruction}

For the function $ f(x) = x^2 - 3x + 5 $ evaluate

\begin{enumerate}[label=(\alph*)]
  \item $ f(1) $
  \item $ f(a + h) $
\end{enumerate}

\subsubsection*{Solution}

\begin{enumerate}[label=(\alph*)]
  \item $ f(1) = 1 ^ 2 - 3 \cdot 1 + 5 = 1 - 3 + 5 = 3 $.
  \item $ f(a + h) = (a + h)^2 - 3(a + h) + 5 = a^2 + 2ah + h^2 - 3a - 3h + 5 $.
\end{enumerate}

\subsubsection*{Answer}

\begin{enumerate}[label=(\alph*)]
  \item $ f(1) = 3 $.
  \item $ f(a + h) = a^2 + 2ah + h^2 - 3a - 3h + 5 $.
\end{enumerate}

\subsection*{Checkpoint 1.2: Finding Domain and Range}

\subsubsection*{Instruction}

Find the domain and range for $ f(x) = \sqrt{4 - 2x} + 5 $.

\subsubsection*{Solution}

\begin{enumerate}[label=\roman*]
  \item To find the domain of $ f $, we need the expression $ 4 - 2x \ge 0 $, due to that real negative numbers do not have a square root. Solving this inequality, we conclude that the domain is $ \{ x \mid x \le 2 \} $.
  \item To find the range of $ f $, we note that since $ \sqrt{4 - 2x} \ge 0 $, it follows that $ f(x) = \sqrt{4 - 2x} + 5 \ge 5 $. Therefore, the range of $ f $ must be a subset of the set $ \{ y \mid y \ge 5 \} $.

    To show that every element in this set is in the range of $ f $, we need to show that for all $ y $ in this set, there exists a real number $ x $ in the domain such that $ f(x) = y $. Let $ y \ge 5 $. Then, $ f(x) = y $ if and only if
    $$ \phantom{.}
    \sqrt{4 - 2x} + 5 = y
    .$$
    Solving this equation for $ x $, we see that $ x $ must solve the equation
    $$ \phantom{.}
    \sqrt{4 - 2x}= y - 5
    .$$
    Since $ y \ge 5 $, such an $ x $ could exist. Squaring both sides of the above equation we have
    $$ \phantom{.}
    4-2x = (y - 5)^2
    .$$
    Therefore we need
    $$ \phantom{,}
    -2x= (y - 5)^2 - 4
    ,$$
    which implies
    $$ \phantom{,}
    x= 2 -\frac{(y - 5)^2}{2}
    .$$
    We just need to verify that $ x $ is in the domain of $ f $. Since the domain of $ f $ consists of all real numbers less or equal to $ 2 $, and
    $$ \phantom{,}
    2 -\frac{(y - 5)^2}{2} \le 2
    ,$$
    there does exist an $ x $ in the domain of $ f $. We conclude that the range of $ f $ is $ \{ y \mid y \ge 5 \} $.
\end{enumerate}

\subsubsection*{Answer}

Domain $ = \{ x \mid x \le 2 \} $, range $ = \{ y \mid y \ge 5 \} $.

\subsection*{Checkpoint 1.3: Finding Zeroes}

\subsubsection*{Instruction}

Find the zeroes of $ f(x) = x^3 - 5x^2 + 6x $.

\subsubsection*{Solution}

The zeroes of a function are the values of $ x $ where $ f(x) = 0 $. To find the zeroes, we need to solve
$$ \phantom{.}
f(x) = x^3 - 5x^2 + 6x = 0
.$$
Factor out $ x $
$$ \phantom{.}
f(x) = x(x^2 - 5x + 6) = 0
.$$
We can continue factoring by pure inspection, with the goal of finding a pair of numbers that add up to $ -5 $ and whose product is $ 6 $. This pair of numbers turns out to be $ -2 $ and $ -3 $, leading to the factoring
$$ \phantom{.}
f(x) = x(x - 2)(x - 3) = 0
.$$
From the above complete factoring of $ f $, we conclude that there are three zeroes when $ x $ is 0, 2, and 3.

\subsubsection{Answer}

$ x = 0, 2, 3 $.

\subsection*{Checkpoint 1.4: Combining Functions Using Mathematical Operations}

\subsubsection{Instruction}

For $ f(x) = x^2 + 3 $ and $ g(x) = 2x - 5 $, find $ (f/g)(x) $ and state its domain.

\subsubsection{Solution}

To find $ (f/g)(x) $ we write the function with the quotient operator
$$ \phantom{.}
\frac{f}{g}(x) = \frac{x^2 + 3}{2x - 5}
.$$
The domain of this function is $ \{ x \mid x \ne \frac{5}{2} \} $.

\subsubsection{Answer}

$ \frac{f}{g}(x) = \frac{x^2 + 3}{2x - 5} $. The domain is $ \{ x \mid x \ne \frac{5}{2} \} $.

\subsection*{Checkpoint 1.5: Compositions of Functions}

\subsubsection{Instruction}

Let $ f(x) = 2 - 5x $. Let $ g(x) = \sqrt{x} $. Find $ (f \circ g)(x) $.

\subsubsection{Solution}

$ (f \circ g)(x) = f(g(x)) = f(\sqrt{x}) = 2 - 5\sqrt{x} $.

\subsubsection{Answer}

$ (f \circ g)(x) = 2 - 5\sqrt{x} $.

\subsection*{Checkpoint 1.6: Application Involving a Composite Function}

\subsubsection{Instruction}

If items are on sale for 10\% off their original price, and a customer has a coupon for an additional 30\% off, what will be the final price for an item that is originally x dollars, after applying the coupon to the sale price?

\subsubsection{Solution}

Since the sale price 10\% off the original price, if an item is $ x $ dollars, its sale price is given by
$$ \phantom{.}
f(x) = 0.90x
.$$
Since the coupon entitles an individual to 30\% off the price of any item, if an item is $ y $ dollars, the price after applying the coupon, is given by
$$ \phantom{.}
g(y) = 0.70y
.$$
Therefore, if the price is originally $ x $ dollars, its price after applying the coupon to the sale price will be
$$ \phantom{.}
(g \circ f)(x) = g(f(x)) = (0.70)0.90x = 0.63x.
.$$

\subsubsection{Answer}

$ (g \circ f)(x) = 0.63x $.


\subsection*{Exercise 1.1.1}

\subsubsection{Instruction}

Assuming the relation in table \ref{table:exercise-1.1.1}.
\begin{enumerate}[label=(\alph*)]
  \item Determine the domain and the range of the relation.
  \item State whether the relation is a function.
\end{enumerate}

\begin{table}[ht]
  \centering
  \begin{tabular}{ c | r r r r r r r }
    \hline
    $ x $ & $ -3 $ & $ -2 $ & $ -1 $ & $ 0 $ & $ 1 $ & $ 2 $ & $ 3 $ \\
    \hline
    $ y $ & $ 9 $ & $ 4 $ & $ 1 $ & $ 0 $ & $ 1 $ & $ 4 $ & $ 9 $ \\
    \hline
  \end{tabular}
  \caption{Relation between $ x $ and $ y $ in exercise 1.1.1}
  \label{table:exercise-1.1.1}
\end{table}

\subsubsection{Solution}

\begin{enumerate}[label=(\alph*)]
  \item The domain of the relation is the set of unique $ x $ values,
    $$ \phantom{.}
    \{ -3, -2, -1, 0, 1, 2, 3 \}
    .$$
    The range of the relation is the set of unique $ y $ values,
    $$ \phantom{.}
    \{ 0, 1, 4, 9 \}
    .$$
  \item This relation is a function, each input is a assigned to exactly one output.
\end{enumerate}

\subsubsection{Answer}

\begin{enumerate}[label=(\alph*)]
  \item Domain = $ \{ -3, -2, -1, 0, 1, 2, 3 \} $, range = $ \{ 0, 1, 4, 9 \} $.
  \item Yes, a function.
\end{enumerate}

\subsection*{Exercise 1.1.2}

\subsubsection{Instruction}

Assuming the relation in table \ref{table:exercise-1.1.2}.
\begin{enumerate}[label=(\alph*)]
  \item Determine the domain and the range of the relation.
  \item State whether the relation is a function.
\end{enumerate}

\begin{table}[ht]
  \centering
  \begin{tabular}{ c | r r r r r r r }
    \hline
    $ x $ & $ -3 $ & $ -2 $ & $ -1 $ & $ 0 $ & $ 1 $ & $ 2 $ & $ 3 $ \\
    \hline
    $ y $ & $ -2 $ & $ -8 $ & $ -1 $ & $ 0 $ & $ 1 $ & $ 8 $ & $ -2 $ \\
    \hline
  \end{tabular}
  \caption{Relation between $ x $ and $ y $ in exercise 1.1.2}
  \label{table:exercise-1.1.2}
\end{table}

\subsubsection{Solution}

\begin{enumerate}[label=(\alph*)]
  \item The domain of the relation is the set of unique $ x $ values,
    $$ \phantom{.}
    \{ -3, -2, -1, 0, 1, 2, 3 \}
    .$$
    The range of the relation is the set of unique $ y $ values,
    $$ \phantom{.}
    \{ -8, -2, -1, 0, 1, 8\}
    .$$
  \item This relation is a function, each input is a assigned to exactly one output.
\end{enumerate}

\subsubsection{Answer}

\begin{enumerate}[label=(\alph*)]
  \item Domain = $ \{ -3, -2, -1, 0, 1, 2, 3 \} $, range = $ \{ -8, -2, -1, 0, 1, 8\} $.
  \item Yes, a function.
\end{enumerate}

\subsection*{Exercise 1.1.3}

\subsubsection{Instruction}

Assuming the relation in table \ref{table:exercise-1.1.3}.
\begin{enumerate}[label=(\alph*)]
  \item Determine the domain and the range of the relation.
  \item State whether the relation is a function.
\end{enumerate}

\begin{table}[ht]
  \centering
  \begin{tabular}{ c | r r r r r r r }
    \hline
    $ x $ & $ 1 $ & $ 2 $ & $ 3 $ & $ 0 $ & $ 1 $ & $ 2 $ & $ 3 $ \\
    \hline
    $ y $ & $ -3 $ & $ -2 $ & $ -1 $ & $ 0 $ & $ 1 $ & $ 2 $ & $ 3 $ \\
    \hline
  \end{tabular}
  \caption{Relation between $ x $ and $ y $ in exercise 1.1.3}
  \label{table:exercise-1.1.3}
\end{table}

\subsubsection{Solution}

\begin{enumerate}[label=(\alph*)]
  \item The domain of the relation is the set of unique $ x $ values,
    $$ \phantom{.}
    \{ 0, 1, 2, 3 \}
    .$$
    The range of the relation is the set of unique $ y $ values,
    $$ \phantom{.}
    \{ -3, -2, -1, 0, 1, 2, 3 \}
    .$$
  \item This relation is not a function, each input is not assigned to exactly one output. Take for example $ x = 1 $ that can cause both $ y = -3 $ and $ y = 1 $.
\end{enumerate}

\subsubsection{Answer}

\begin{enumerate}[label=(\alph*)]
  \item Domain = $ \{ 0, 1, 2, 3 \} $, range = $ \{ -3, -2, -1, 0, 1, 2, 3 \} $.
  \item No, not a function.
\end{enumerate}

\subsection*{Exercise 1.1.4}

\subsubsection{Instruction}

Assuming the relation in table \ref{table:exercise-1.1.4}.
\begin{enumerate}[label=(\alph*)]
  \item Determine the domain and the range of the relation.
  \item State whether the relation is a function.
\end{enumerate}

\begin{table}[ht]
  \centering
  \begin{tabular}{ c | r r r r r r r }
    \hline
    $ x $ & $ 1 $ & $ 2 $ & $ 3 $ & $ 4 $ & $ 5 $ & $ 6 $ & $ 7 $ \\
    \hline
    $ y $ & $ 1 $ & $ 1 $ & $ 1 $ & $ 1 $ & $ 1 $ & $ 1 $ & $ 1 $ \\
    \hline
  \end{tabular}
  \caption{Relation between $ x $ and $ y $ in exercise 1.1.4}
  \label{table:exercise-1.1.4}
\end{table}

\subsubsection{Solution}

\begin{enumerate}[label=(\alph*)]
  \item The domain of the relation is the set of unique $ x $ values,
    $$ \phantom{.}
    \{ 1, 2, 3, 4, 5, 6, 7 \}
    .$$
    The range of the relation is the set of unique $ y $ values,
    $$ \phantom{.}
    \{ 1 \}
    .$$
  \item This relation is a function, each input is a assigned to exactly one output.
\end{enumerate}

\subsubsection{Answer}

\begin{enumerate}[label=(\alph*)]
  \item Domain = $  \{ 1, 2, 3, 4, 5, 6, 7 \} $, range = $ \{ 1 \} $.
  \item Yes, a function.
\end{enumerate}

\subsection*{Exercise 1.1.5}

\subsubsection{Instruction}

Assuming the relation in table \ref{table:exercise-1.1.5}.
\begin{enumerate}[label=(\alph*)]
  \item Determine the domain and the range of the relation.
  \item State whether the relation is a function.
\end{enumerate}

\begin{table}[ht]
  \centering
  \begin{tabular}{ c | r r r r r r r }
    \hline
    $ x $ & $ 3 $ & $ 5 $ & $ 8 $ & $ 10 $ & $ 15 $ & $ 21 $ & $ 33 $ \\
    \hline
    $ y $ & $ 3 $ & $ 2 $ & $ 1 $ & $ 0 $ & $ 1 $ & $ 2 $ & $ 3 $ \\
    \hline
  \end{tabular}
  \caption{Relation between $ x $ and $ y $ in exercise 1.1.5}
  \label{table:exercise-1.1.5}
\end{table}

\subsubsection{Solution}

\begin{enumerate}[label=(\alph*)]
  \item The domain of the relation is the set of unique $ x $ values,
    $$ \phantom{.}
    \{ 3, 5, 8, 10, 15, 21, 33 \}
    .$$
    The range of the relation is the set of unique $ y $ values,
    $$ \phantom{.}
    \{ 0, 1, 2, 3\}
    .$$
  \item This relation is a function, each input is a assigned to exactly one output.
\end{enumerate}

\subsubsection{Answer}

\begin{enumerate}[label=(\alph*)]
  \item Domain = $  \{ 3, 5, 8, 10, 15, 21, 33 \} $, range = $ \{ 0, 1, 2, 3\} $.
  \item Yes, a function.
\end{enumerate}

\subsection*{Exercise 1.1.6}

\subsubsection{Instruction}

Assuming the relation in table \ref{table:exercise-1.1.6}.
\begin{enumerate}[label=(\alph*)]
  \item Determine the domain and the range of the relation.
  \item State whether the relation is a function.
\end{enumerate}

\begin{table}[ht]
  \centering
  \begin{tabular}{ c | r r r r r r r }
    \hline
    $ x $ & $ -7 $ & $ -2 $ & $ -2 $ & $ 0 $ & $ 1 $ & $ 3 $ & $ 6 $ \\
    \hline
    $ y $ & $ 11 $ & $ 5 $ & $ 1 $ & $ -1 $ & $ -2 $ & $ 4 $ & $ 11 $ \\
    \hline
  \end{tabular}
  \caption{Relation between $ x $ and $ y $ in exercise 1.1.6}
  \label{table:exercise-1.1.6}
\end{table}

\subsubsection{Solution}

\begin{enumerate}[label=(\alph*)]
  \item The domain of the relation is the set of unique $ x $ values,
    $$ \phantom{.}
    \{ -7, -2, 0, 1, 3, 6 \}
    .$$
    The range of the relation is the set of unique $ y $ values,
    $$ \phantom{.}
    \{ -2, -1, 1, 4, 5, 11\}
    .$$
  \item This relation is not a function, each input is not assigned to exactly one output. See $ x = -2, $ that can cause both $ y = 1 $ and $ y = 5 $.
\end{enumerate}

\subsubsection{Answer}

\begin{enumerate}[label=(\alph*)]
  \item Domain = $  \{ -7, -2, 0, 1, 3, 6 \} $, range = $ \{ -2, -1, 1, 4, 5, 11\} $.
  \item No, not a function.
\end{enumerate}

\subsection*{Exercise 1.1.7}

\subsubsection{Instruction}

Find the below values for the function $ f(x) = 5x - 2 $, if they exist, then simplify.
\begin{enumerate}[label=(\alph*)]
  \item $ f(0) $
  \item $ f(1) $
  \item $ f(3) $
  \item $ f(-x) $
  \item $ f(a) $
  \item $ f(a + h) $
\end{enumerate}

\subsubsection{Solution}

\begin{enumerate}[label=(\alph*)]
  \item $ f(0) = 5 \cdot 0 - 2 = 0 - 2 = -2 $.
  \item $ f(1) = 5 \cdot 1 - 2 = 5 - 2 = 3 $.
  \item $ f(2) = 5 \cdot 3 - 2 = 15 - 2 = 13 $.
  \item $ f(-x) = 5(-x) - 2 = -5x - 2 $.
  \item $ f(a) = 5a - 2 $.
  \item $ f(a + h) = 5(a + h) - 2 = 5a + 5h - 2 $.
\end{enumerate}

\subsubsection{Answer}

\begin{enumerate}[label=(\alph*)]
  \item $ -2 $.
  \item $ 3 $.
  \item $ 13 $.
  \item $ -5x - 2 $.
  \item $ 5a - 2 $.
  \item $ 5a + 5h - 2$.
\end{enumerate}

\subsection*{Exercise 1.1.8}

\subsubsection{Instruction}

Find the below values for the function $ f(x) = 4x^2 - 3x + 1 $, if they exist, then simplify.
\begin{enumerate}[label=(\alph*)]
  \item $ f(0) $
  \item $ f(1) $
  \item $ f(3) $
  \item $ f(-x) $
  \item $ f(a) $
  \item $ f(a + h) $
\end{enumerate}

\subsubsection{Solution}

\begin{enumerate}[label=(\alph*)]
  \item $ f(0) = 4 \cdot 0^2 - 3 \cdot 0 + 1 = 4 \cdot 0 - 0 + 1 = 0 - 0 + 1 = 1 $.
  \item $ f(1) = 4 \cdot 1^2 - 3 \cdot 1 + 1 = 4 \cdot 1 - 3 + 1 = 4 - 3 + 1 = 2 $.
  \item $ f(3) = 4 \cdot 3^2 - 3 \cdot 3 + 1 = 4 \cdot 9 - 9 + 1 = 36 - 9 + 1 = 28 $.
  \item $ f(-x) = 4(-x)^2 - 3(-x) + 1 = 4x^2 + 3x + 1 $.
  \item $ f(a) = 4a^2 - 3a + 1 $.
  \item
    \(
      \begin{aligned}[t]
        f(a + h) &= 4(a + h)^2 - 3(a + h) + 1 \\
        &= 4(a^2 + 2ah + h^2) - 3a - 3h + 1 \\
        &= 4a^2 + 4h^2 + 8ah -3a - 3h + 1.
      \end{aligned}
    \)

\end{enumerate}

\subsubsection{Answer}

\begin{enumerate}[label=(\alph*)]
  \item $ 1 $.
  \item $ 2 $.
  \item $ 28 $.
  \item $ 4x^2 + 3x + 1 $.
  \item $ 4a^2 - 3a + 1 $.
  \item $ 4a^2 + 4h^2 + 8ah -3a - 3h + 1 $.
\end{enumerate}

\subsection*{Exercise 1.1.15}

Find the domain, range, and all zeros/intercepts, if any, of the function $ g(x) = \sqrt{8x - 1} $.

\subsubsection{Solution}

\begin{enumerate}[label=\roman*]

  \item
    The domain of the square root function is $ \left[ 0, \infty \right) $, which implies $ 8x - 1 \ge 0 $. Solving for $ x $ gives $ x \ge \frac{1}{8} $.

  \item
    To find the range of $ g $, we note that $ \sqrt{8x - 1} \ge 0 $. Therefore, the range of $ g $ must be a subset of the set $ \{y \mid y \ge 0\} $. To show that every element in this set is in the range of $ g $, we need to show that for a given $ y $ in this set, there exists a real number $ x $ in the domain such that $ g(x) = y $.

    Let $ y \ge 0 $. Then $ g(x) = y $ if and only if
    $$ \phantom{.}
    \sqrt{8x - 1} = y
    .$$

    We are interested in $ x $, and will solve this equation for $ x $. Since $ y \ge 0 $ such an $ x $ could exist. Squaring both sides of this equation, we have
    $$ \phantom{.}
    {8x - 1} = y^2
    .$$
    Therefore, we need
    $$ \phantom{.}
    8x = y^2 + 1
    ,$$
    which implies
    $$ \phantom{.}
    x = \frac{y^2 + 1}{8}
    .$$
    We just need to verify that $ x $ is in the domain of $ g $. Since the domain of $ g $ consists of all real numbers greater than or equal to $ 1 / 8 $, and
    $$ \phantom{.}
    \frac{y^2 + 1}{8} \ge \frac{1}{8}
    ,$$
    there does exist an $ x $ in the domain of $ g $. We conclude that the range of $ g $ is $ \{ y \mid y \ge 0 \} $.

  \item
    To find the zeroes, solve $ g(x) = \sqrt{8x - 1}0 $. We discover that $ g $ have one zero at $ x = -1/8 $.

  \item The y-intercept is given by $ (0, g(0)) $. Since $ x = 0 $ isn't in the domain of $ g $, it follows that that there aren't any intercepts.

\end{enumerate}

\subsubsection{Answer}

Domain = $ x \ge \frac{1}{8} $, range = $ \{ y \mid y \ge 0 \} $, zeroes $ x = -1/8 $, no intercepts.

\subsection*{Exercise 1.1.23}

Sketch the graph for the function $ f(x) = 3x - 6 $ with the aid of table \ref{table:exercise-1.1.23}.

\begin{table}[ht]
  \centering
  \begin{tabular}{ c | r r r r r r r }
    \hline
    $ x $ & $ -3 $ & $ -2 $ & $ -1 $ & $ 0 $ & $ 1 $ & $ 2 $ & $ 3 $ \\
    \hline
    $ y $ & $ -15 $ & $ -12 $ & $ -9 $ & $ -6 $ & $ -3 $ & $ 0 $ & $ 3 $ \\
    \hline
  \end{tabular}
  \caption{Relation between $ x $ and $ y $ in exercise 1.1.23}
  \label{table:exercise-1.1.23}
\end{table}

\subsubsection{Solution}

Begin by sketching the axes. We choose the same scale on both axes to not distort the graph. We choose the range for both axes to be -15 to 15, allowing us to plot all the points from table \ref{table:exercise-1.1.23}, see figure \ref{figure:exercise-1.1.23-figure-1}.

\begin{figure}[H]
  \centering
  \begin{tikzpicture}
    \centering
    \begin{axis}[
        axis x line = center,
        axis y line = center,
        xlabel = \( x \),
        ylabel = \( y \),
        xmin = -17.5, xmax = 17.5,
        ymin = -17.5, ymax = 17.5,
        xtick distance = 5,
        ytick distance = 5,
      ]
    \end{axis}
  \end{tikzpicture}
  \caption{Empty graph with just the axes}
  \label{figure:exercise-1.1.23-figure-1}
\end{figure}

After having sketched the axes we add markers based on the data in table \ref{table:exercise-1.1.23}, see figure \ref{figure:exercise-1.1.23-figure-2}.

\begin{figure}[H]
  \centering
  \begin{tikzpicture}
    \centering
    \begin{axis}[
        axis x line = center,
        axis y line = center,
        xlabel = \( x \),
        ylabel = \( y \),
        xmin = -17.5, xmax = 17.5,
        ymin = -17.5, ymax = 17.5,
        xtick distance = 5,
        ytick distance = 5,
      ]
      \addplot[
        only marks,
        mark = *,
      ]
      coordinates {
        (-3,-15)(-2,-12)(-1,-9)(0,-6)(1,-3)(2,0)(3,3)
      };
    \end{axis}
  \end{tikzpicture}
  \caption{Graph with added markers}
  \label{figure:exercise-1.1.23-figure-2}
\end{figure}

We then connect the markers with line segments. In this particular case the result will be a single straight line so we can use a ruler when sketching, , see figure \ref{figure:exercise-1.1.23-figure-3}.

\begin{figure}[H]
  \centering
  \begin{tikzpicture}
    \centering
    \begin{axis}[
        axis x line = center,
        axis y line = center,
        xlabel = \( x \),
        ylabel = \( y \),
        xmin = -17.5, xmax = 17.5,
        ymin = -17.5, ymax = 17.5,
        xtick distance = 5,
        ytick distance = 5,
      ]
      \addplot[
        only marks,
        mark = *,
      ]
      coordinates {
        (-3,-15)(-2,-12)(-1,-9)(0,-6)(1,-3)(2,0)(3,3)
      };
      \addplot [
        domain = -17.5:17.5,
        samples = 100,
      ]
      {3 * x - 6};
    \end{axis}
  \end{tikzpicture}
  \caption{Graph with connected markers}
  \label{figure:exercise-1.1.23-figure-3}
\end{figure}

\subsubsection*{Answer}

\begin{figure}[H]
  \centering
  \begin{tikzpicture}
    \centering
    \begin{axis}[
        axis x line = center,
        axis y line = center,
        xlabel = \( x \),
        ylabel = \( y \),
        xmin = -17.5, xmax = 17.5,
        ymin = -17.5, ymax = 17.5,
        xtick distance = 5,
        ytick distance = 5,
        legend pos = south east,
      ]
      \addplot [
        domain = -17.5:17.5,
        samples = 100,
      ]
      {3 * x - 6};
      \addlegendentry{\( 3x - 6 \)}
    \end{axis}
  \end{tikzpicture}
  \caption{Answer to exercise 1.1.23}
\end{figure}

\subsection*{Exercise 1.1.29}

Use the vertical line test to determine whether the graph in figure \ref{figure:exercise-1.1.29} represent a function. Assume that the graph continues at both ends beyond the given grid. If the graph represents a function, then determine the following for the graph:

\begin{enumerate}[label=(\alph*)]
  \item Domain and range
  \item $ x $-intercept, if any (estimate where necessary)
  \item $ y $-intercept, if any (estimate where necessary)
  \item The intervals for which the function is increasing
  \item The intervals for which the function is decreasing
  \item The intervals for which the function is constant
  \item Symmetry about any axis and/or the origin
  \item Whether the function is even, odd, or neither
\end{enumerate}

\begin{figure}[H]
  \centering
  \begin{tikzpicture}
    \centering
    \begin{axis}[
        axis x line = center,
        axis y line = center,
        xlabel = \( x \),
        ylabel = \( y \),
        xmin = -5.5, xmax = 5.5,
        ymin = -5.5, ymax = 5.5,
        xtick distance = 1,
        ytick distance = 1,
        width = .80\textwidth
      ]
      \addplot [
        domain = -2:2,
        samples = 200,
      ]
      {x^4 - 2*x^2 + 1};
    \end{axis}
  \end{tikzpicture}
  \caption{Graph for exercise 1.1.29}
  \label{figure:exercise-1.1.29}
\end{figure}

\subsubsection{Solution}

The graph in figure \ref{figure:exercise-1.1.29} do represent a function because every vertical line that may be drawn intersects the graph no more than once. See figure \ref{figure:exercise-1.1.29-vertical-line-test} for an example of a vertical line with one intersection of the graph. We could slide this line over the entire graph and there would always only be at most one intersection.

\begin{figure}[H]
  \centering
  \begin{tikzpicture}
    \centering
    \begin{axis}[
        axis x line = center,
        axis y line = center,
        xlabel = \( x \),
        ylabel = \( y \),
        xmin = -5.5, xmax = 5.5,
        ymin = -5.5, ymax = 5.5,
        xtick distance = 1,
        ytick distance = 1,
        width = .80\textwidth
      ]
      % Plot the graph.
      \addplot [
        domain = -2:2,
        samples = 100,
      ]
      {x^4 - 2*x^2 + 1};
      % Plot a vertical line.
      \addplot[
        no marks,
      ]
      coordinates{(0.5,5.5) (0.5,-5.5)};
      % Mark the intersection between the plot and the vertical line.
      \addplot[
        mark = *,
      ]
      coordinates{(0.5,0.5625)};
    \end{axis}
  \end{tikzpicture}
  \caption{Vertical line test illustration}
  \label{figure:exercise-1.1.29-vertical-line-test}
\end{figure}

\begin{enumerate}[label=(\alph*)]
  \item
    \begin{enumerate}[label=\roman*]
      \item The function seems to grow rapidly as $ x $ goes towards $ \pm\infty $, but there will still always be a $ y $ value. We conclude that the domain is all real numbers.
      \item $ y $ is always greater or equal to $ 0 $, this is the range.
    \end{enumerate}
  \item $ y $ is zero for $ x = -1 $, and $ x = 1 $, these are the $ x $-intercepts.
  \item The $ y $-intercept is $ y = 1 $.
  \item The function is increasing for the intervals $ -1 < x < 0 $ and $ 1 < x < \infty $.
  \item The function is decreasing for the intervals $ -\infty < x < -1 $ and $ 0 < x < 1 $.
  \item The function changes from decreasing/increasing when $ x $ is $ -1 $, $ 0 $, and $ 1 $, but there are no intervals for which the function is constant.
  \item $ (-x, y) $ is on the graph whenever $ (x, y) $ is on the graph, in other words the function is symmetric around the $ y $-axis.
  \item The function is not odd because $ f(-x) \neq -f(x) $ for all $ x $ in the domain. Take for example $ x = 0.5 $ for which $ f(-x) \approx 0.6 $ and $ -f(x) \approx - 0.6 $.

    The function is even because $ f(-x) = f(x) $ for all x. Take for example  $ x = 0.5 $ for which $ f(-x) \approx 0.6 $ and $ f(x) \approx 0.6 $.
\end{enumerate}

\subsubsection*{Answer}

Graph represents a function.

\begin{enumerate}[label=(\alph*)]
  \item Domain: all real numbers, range: $ y \ge 0 $.
  \item $ x = -1 $ and $ x = 1 $.
  \item $ y = 1 $.
  \item $ -1 < x < 0 $ and $ 1 < x < \infty $.
  \item $ -\infty < x < -1 $ and $ 0 < x < 1 $.
  \item Not constant.
  \item $ y $-axis.
  \item Even.
\end{enumerate}

\subsection*{Exercise 1.1.37}

\subsubsection*{Instruction}

For the pair of functions $ f(x) = x - 8 $ and $ g(x) = 5x^2 $, find the below new listed functions. Also determine the domain for each of these new functions.

\begin{enumerate}[label = (\alph*)]
  \item $ f + g $
  \item $ f - g $
  \item $ f \cdot g $
  \item $ f / g $
\end{enumerate}

\subsubsection*{Solution}

\begin{enumerate}[label = (\alph*)]
  \item
    Add the two given functions to form the requested function,
    $$ \phantom{.}
    f + g = x - 8 + 5x^2 = 5x^2 + x - 8
    .$$
    The domain of the above new function is all real numbers.
  \item
    Subtract the two given functions to form the requested function,
    $$ \phantom{.}
    f - g = x - 8 - 5x^2 = -5x^2 + x - 8
    .$$
    The domain of the above new function is all real numbers.
  \item
    Multiply the two given functions to form the requested function,
    $$ \phantom{.}
    f \cdot g = (x - 8)5x^2 = 5x^3 - 8x^2
    .$$
    The domain of the above new function is all real numbers.
  \item
    Divide the two given functions to form the requested function,
    $$ \phantom{.}
    \frac{f}{g} = \frac{x - 8}{5x^2}
    .$$
    The division is defined except for for $ x = 0 $, the domain is hence $ x \neq 0 $.

\end{enumerate}

\subsubsection*{Answer}

\begin{enumerate}[label = (\alph*)]
  \item
    $ 5x^2 + x - 8 $, domain: all real numbers.
  \item
    $ -5x^2 + x - 8 $, domain: all real numbers.
  \item
    $ 5x^3 - 8x^2 $, domain: all real numbers.
  \item
    $ \frac{x - 8}{5x^2} $, domain: $ x \neq 0 $.
\end{enumerate}

\subsection*{Exercise 1.1.43}

\subsubsection*{Instruction}

For the pair of functions $ f(x) = x + 4 $ and $ g(x) = 4x - 1 $, find the below listed compositions. Simplify the results. Find the domain of each of the results.

\begin{enumerate}[label = (\alph*)]
  \item $ (f \circ g)(x) $
  \item $ (g \circ f)(x) $
\end{enumerate}

\subsubsection*{Solution}

\begin{enumerate}[label = (\alph*)]
  \item
    The composition is given by
    $$ \phantom{.}
    (f \circ g)(x) = f(g(x)) = (4x - 1) + 4 = 4x - 1 + 4 = 4x + 3
    .$$
    The domain of the above composition is all real numbers.
  \item
    The composition is given by
    $$ \phantom{.}
    (g \circ f)(x) = g(f(x)) = 4(x + 4) -1 = 4x + 16 - 1 = 4x + 15
    .$$
    The domain of the above composition is all real numbers.
\end{enumerate}

\subsubsection*{Answer}

\begin{enumerate}[label = (\alph*)]
  \item
    $ 4x + 3 $, domain: all real numbers.
  \item
    $ 4x + 15 $, domain: all real numbers.
\end{enumerate}

\subsection*{Exercise 1.1.49}

\subsubsection*{Instruction}

Table \ref{table:exercise-1.1.49} lists the NBA championship winners for the years 2001 to 2012.

\begin{table}[ht]
  \centering
  \begin{tabular}{ l l }
    \hline
    Year & Winner \\
    \hline
    $ 2001 $ & La Lakers \\
    $ 2002 $ & La Lakers \\
    $ 2003 $ & San Antonio Spurs \\
    $ 2004 $ & Detroit Pistons \\
    $ 2005 $ & San Antonio Spurs \\
    $ 2006 $ & Miami Heat \\
    $ 2007 $ & San Antonio Spurs \\
    $ 2008 $ & Boston Celtics \\
    $ 2009 $ & La Lakers \\
    $ 2010 $ & La Lakers \\
    $ 2011 $ & Dallas Mavericks \\
    $ 2012 $ & Miami Heat \\
    \hline
  \end{tabular}
  \caption{NBA championship winners for the years 2001 to 2012}
  \label{table:exercise-1.1.49}
\end{table}

\begin{enumerate}[label = (\alph*)]
  \item
    Consider the relation in which the domain values are the years 2001 to 2012 and the range is the corresponding winner. Is this relation a function? Explain why or why not.
  \item
    Consider the relation where the domain values are the winners and the range is the corresponding years. Is this relation a function? Explain why or why not.
\end{enumerate}

\subsubsection*{Solution}

\begin{enumerate}[label = (\alph*)]
  \item
    The relation in which the domain values are the years and the range is the corresponding winner is a function because a given year have only one winner. This functions set of inputs is the years 2001 to 2012 and the output is a team name. The rule for assigning each input to exactly one output is defined by table \ref{table:exercise-1.1.49}.
  \item
    The relation where the domain values are the winners and the range is the corresponding years is not a function because there are teams that have won more than once during the years. A function shall have a rule for assigning each input to exactly one output. In this case we cannot deduce exactly one year from just knowing a team name.
\end{enumerate}

\subsubsection*{Answer}

\begin{enumerate}[label = (\alph*)]
  \item
    Yes, a function.
  \item
    No, not a function.
\end{enumerate}

\subsubsection*{Exercise 1.1.51}

\subsubsection*{Instruction}

The volume of a cube depends on the length of the sides $ s $.

\begin{enumerate}[label = (\alph*)]
  \item
    Write a function V(s) for the volume of the cube.
  \item
    Find an interpret $ V(11.8) $.
\end{enumerate}

\subsubsection*{Solution}

\begin{enumerate}[label = (\alph*)]
  \item
    A cube will have sides $ s $ of equal length. The volume is found by multiplying $ s $ three times
    $$ \phantom{.}
    V(s) = s \cdot s \cdot s = s^3
    .$$
  \item
    A cube with the side equal to $ 11.8 $ length units will have the volume
    $$
    V(11.8) = 11.8^3 \approx 1643
    $$
    cubic units.
\end{enumerate}

\subsubsection*{Answer}

\begin{enumerate}[label = (\alph*)]
  \item
    $ V(s) = s^3 $.
  \item
    $ V(11.8) = 11.8^3 \approx 1643 $ cubic units.
\end{enumerate}

\subsubsection*{Exercise 1.1.57}

\subsubsection*{Instruction}

The manager at a skateboard shop pays his workers a monthly salary $ S $ of \$750 plus a commission of \$8.50 for each skateboard they sell.

\begin{enumerate}[label = (\alph*)]
  \item
    Write a function $ y = S(x) $ that models a worker’s monthly salary based on the number of skateboards $ x $ he or she sells.
  \item
    Find the monthly salary when a worker sells 25, 40, or 55 skateboards.
  \item
    Use the INTERSECT feature on a graphing calculator to determine the number of skateboards that must be sold for a worker to earn a monthly income of \$1400. (Hint: Find the intersection of the function and the line $ y = 1400 $.)
\end{enumerate}

\subsubsection*{Solution}

\begin{enumerate}[label = (\alph*)]
  \item
    The workers have a base salary plus a commission based on number of skateboards sales. The function will be the constant base salary plus a product depending $ x $ being number of skateboards sold,
    $$ \phantom{.}
    y = S(x) = 750 + 8.50 \cdot x
    .$$
  \item
    Having the formula from above we can calculate monthly salary for the different amount of skateboards sold,
    $$ \phantom{,}
    S(25) = 750 + 8.50 \cdot 25 = 962.5
    ,$$
    $$ \phantom{,}
    S(40) = 750 + 8.50 \cdot 40 = 1090
    ,$$
    $$ \phantom{.}
    S(55) = 750 + 8.50 \cdot 55 = 1217.5
    .$$
  \item
    Using a graphing calculator to graph our function and the liny $ y = 1400 $ we note that there will be two lines than intersect at the point $ (76.47, 1400) $. We can conclude that a worker will need to sell 77 skateboards to earn \$1400.

\end{enumerate}

\subsubsection*{Answer}

\begin{enumerate}[label = (\alph*)]
  \item
    $ y = S(x) = 750 + 8.50 \cdot x. $
  \item
    \$962.5, \$1090, \$1217.50.
  \item
    77 skateboards.
\end{enumerate}


\section{Basic Classes of Functions}

\subsection*{Exercise 1.1.59}

\subsubsection*{Instruction}

For the line that passes through the pair of points $(-2, 4), (1,1)$.

\begin{enumerate}[label = (\alph*)]
  \item
    Find the slope of the line.
  \item
    Indicate whether the line is increasing, decreasing, horizontal, or vertical.

\end{enumerate}

\subsubsection*{Solution}

\begin{enumerate}[label = (\alph*)]
  \item
    The slope of the line through two points $(x_1, y_1), (x_2, y_2)$ is given by,
    \[ \phantom{.} m = \frac{y_2 - y_1}{x_2 - y_1}. \]
    We substitute the given points to get the slope,
    \[ \phantom{.} m = \frac{1 - 4}{1 - (-2)} = \frac{-3}{1 + 2} = \frac{-3}{3} = -1. \]
  \item
    Since the slope is negative (-1), the line is decreasing.

\end{enumerate}

\subsubsection*{Answer}

\begin{enumerate}[label = (\alph*)]
  \item
    $-1.$
  \item
    Decreasing.
\end{enumerate}


\end{document}
