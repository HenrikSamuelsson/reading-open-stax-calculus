\subsection*{Checkpoint 1.3: Finding Zeroes}

\subsubsection*{Instruction}

Find the zeroes of $ f(x) = x^3 - 5x^2 + 6x $.

\subsubsection*{Solution}

The zeroes of a function are the values of $ x $ where $ f(x) = 0 $. To find the zeroes, we need to solve
$$ \phantom{.}
f(x) = x^3 - 5x^2 + 6x = 0
.$$
Factor out $ x $
$$ \phantom{.}
f(x) = x(x^2 - 5x + 6) = 0
.$$
We can continue factoring by pure inspection, with the goal of finding a pair of numbers that add up to $ -5 $ and whose product is $ 6 $. This pair of numbers turns out to be $ -2 $ and $ -3 $, leading to the factoring
$$ \phantom{.}
f(x) = x(x - 2)(x - 3) = 0
.$$
From the above complete factoring of $ f $, we conclude that there are three zeroes when $ x $ is 0, 2, and 3.

\subsubsection{Answer}

$ x = 0, 2, 3 $.
