\subsection*{Checkpoint 1.16: Transforming a Function}

\subsubsection*{Instruction}

Describe how the function $ f(x) = -(x + 1)^2 - 4 $ using the graph of $ y = x^2 $ and a sequence of transformations.

\subsubsection*{Solution}

We start with the graph of $ y = x^2 $, see figure \ref{figure:exercise-1.1.16-1}.

\begin{figure}[H]
  \centering
  \begin{tikzpicture}
    \centering
    \begin{axis}[
        axis x line = center,
        axis y line = center,
        xlabel = \( x \),
        ylabel = \( y \),
        xmin = -5.5, xmax = 5.5,
        ymin = -5.5, ymax = 5.5,
        xtick distance = 1,
        ytick distance = 1,
        width = .80\textwidth
      ]
      \addplot [
        domain = -3:3,
        samples = 200,
      ]
      {x^2};
      \addlegendentry{\( x^2 \)}
    \end{axis}
  \end{tikzpicture}
  \caption{Starting point}
  \label{figure:exercise-1.1.16-1}
\end{figure}

We shift left by $ 1 $ unit, see figure \ref{figure:exercise-1.1.16-2}.

\begin{figure}[H]
  \centering
  \begin{tikzpicture}
    \centering
    \begin{axis}[
        axis x line = center,
        axis y line = center,
        xlabel = \( x \),
        ylabel = \( y \),
        xmin = -5.5, xmax = 5.5,
        ymin = -5.5, ymax = 5.5,
        xtick distance = 1,
        ytick distance = 1,
        width = .80\textwidth,
      ]
      \addplot [
        domain = -4:3,
        samples = 200,
      ]
      {(x + 1)^2};
      \addlegendentry{\( (x + 1)^2 \)}
    \end{axis}
  \end{tikzpicture}
  \caption{Shift left by $ 1 $}
  \label{figure:exercise-1.1.16-2}
\end{figure}

We apply a factor of $ - 1 $, making the graph reflected, see figure \ref{figure:exercise-1.1.16-3}.

\begin{figure}[H]
  \centering
  \begin{tikzpicture}
    \centering
    \begin{axis}[
        axis x line = center,
        axis y line = center,
        xlabel = \( x \),
        ylabel = \( y \),
        xmin = -5.5, xmax = 5.5,
        ymin = -5.5, ymax = 5.5,
        xtick distance = 1,
        ytick distance = 1,
        width = .80\textwidth,
      ]
      \addplot [
        domain = -4:3,
        samples = 200,
      ]
    {-(x + 1)^2)};
    \addlegendentry{\( -(x + 1)^2 \)}
  \end{axis}
\end{tikzpicture}
\caption{Reflect the graph in the $ x $-axis}
\label{figure:exercise-1.1.16-3}
\end{figure}

We shift down by $ 4 $ units, see figure \ref{figure:exercise-1.1.16-4}.

\begin{figure}[H]
\centering
\begin{tikzpicture}
  \centering
  \begin{axis}[
      axis x line = center,
      axis y line = center,
      xlabel = \( x \),
      ylabel = \( y \),
      xmin = -5.5, xmax = 5.5,
      ymin = -5.5, ymax = 5.5,
      xtick distance = 1,
      ytick distance = 1,
      width = .80\textwidth,
    ]
    \addplot [
      domain = -5.5:5.5,
      samples = 200,
    ]
    {-(x + 1)^2 - 4};
    \addlegendentry{\( -(x + 1)^2 - 4 \)}
  \end{axis}
\end{tikzpicture}
\caption{Shift down by $ 4 $}
\label{figure:exercise-1.1.16-4}
\end{figure}

We have now applied all needed transformations.

\subsubsection*{Answer}

Shift the graph $ y = x^2 $ to the left $ 1 $unit, reflect about the $ x $
-axis, then shift down $ 4 $ units.
