\subsection*{Checkpoint 1.10: Graphing Polynomial Functions}

\subsubsection*{Instruction}

Consider the quadratic function $ f(x) = 3x^2 -6x + 2 $.

\begin{enumerate}[label = (\alph*)]
  \item
    Find the zeroes of $ f $.
  \item
    Doest the parabola open upward or downward?
  \item
    Sketch a graph of $ f $.
\end{enumerate}

\subsubsection*{Solution}

\begin{enumerate}[label = (\alph*)]
  \item
    We find the zeroes of $ f $ using the quadratic function. In this case we have $ a = 3 $, $ b = -6 $, $ c = 2 $. The two zeroes are
    $$ \phantom{.}
    x = \frac{-(-6) \pm \sqrt{(-6)^2 - 4 \cdot 3 \cdot 2}}{2 \cdot 3} = \frac{6 \pm 2\sqrt{3}}{6} = \frac{3 \pm \sqrt{3}}{3} = 1 \pm \frac{\sqrt{3}}{3}
    .$$
    Using an calculator we can find the alternate form $ x_1 \approx 1.58 $, $ x_2 \approx 0.423 $.
  \item
    We have an quadratic function on the form $ f(x) = ax^2 + bx + c $. The plot for this type of function will be a parabola. If $ a > 0 $, then $ f(x) \to \infty $ as $ x \to \infty $ and $ f(x) \to -\infty $ as $ x \to -\infty $. This is due to that $ x^2 $ will eventually start to dominate as $ x $ grows, the other part of the function will not matter. This leads to that the parabola will open upwards for $ a > 0 $. In this case we have $ a = 3 $, which is greater than zero. We conclude that the parabola will open upward.
  \item
    We can manually sketch the parabola by first calculating points $ (x, f(x)) $ for some different $ x $ values around to the zeroes calculated in part a. Then plot these points. Finally connect the points with a parabola shape, remembering from part b that the parabola will open upward.

    Another way is to use graphing tool, can be a calculator or a computer software, see figure \ref{figure:checkpoint-1.10} for an example of the result.

    \begin{figure}[H]
      \centering
      \begin{tikzpicture}
        \centering
        \begin{axis}[
            axis x line = center,
            axis y line = center,
            xlabel = \( x \),
            ylabel = \( y \),
            xmin = -4.5, xmax = 4.5,
            ymin = -4.5, ymax = 4.5,
            xtick distance = 1,
            ytick distance = 1,
            width = .80\textwidth,
            legend pos = south east,
          ]
          \addplot [
            domain = -2:3,
            samples = 200,
          ]
          {3 * x^2 -6 * x + 2};
          \addlegendentry{\( 3x^2 -6x + 2 \)}
        \end{axis}
      \end{tikzpicture}
      \caption{Graph of the quadratic function in checkpoint 1.10}
      \label{figure:checkpoint-1.10}
    \end{figure}
\end{enumerate}

\subsubsection{Answer}

\begin{enumerate}[label = (\alph*)]
  \item
    The zeroes are $ 1 \pm {\sqrt{3}}/3 $.
  \item
    The parabola opens upward.
  \item See the graph in figure \ref{figure:checkpoint-1.10}.
\end{enumerate}
