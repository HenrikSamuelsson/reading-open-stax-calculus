\subsection*{Exercise 1.1.67}

\subsubsection*{Instruction}

Assuming a line with slope $ = -6 $, passing through the point $ (1, 3) $. Write the equation for this line in slope-intercept form.

\subsubsection*{Solution}

We have the slope and one point on the line, this enables to write the equation for the line in point-slope form, which in general is $ y - y_1 = m(x - x_1) $. In this case we get

\[ \phantom{.} y - 3 = -6(x - 1) \text{.} \]

Then to get the slope-intercept form we simply solve the above equation for y,

\[ \phantom{.} y = -6(x - 1) + 3 = -6x + 6 + 3 = -6x + 9 \text{.} \]

\subsubsection*{Answer}

y = -6x + 9.
