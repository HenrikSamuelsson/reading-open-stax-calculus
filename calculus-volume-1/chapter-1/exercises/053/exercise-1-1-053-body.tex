\subsection*{Exercise 1.1.53}

\subsubsection*{Instruction}

A vehicle has a 20-gal tank and gets 15 mpg. The number of miles $ N $ that can be driven depends on the amount of gas $ x $ in the tank.

\begin{enumerate}[label = (\alph*)]
  \item
    Write a formula that models the situation.
  \item
    Determine the number of miles the vehicle can travel on (i) a full tank of gas and (ii) 3/4 of a tank of gas.
  \item
    Determine the domain and range of the function.
  \item
    Determine how many times the driver had to stop for gas if she has driven a total 578 miles.
\end{enumerate}

\subsubsection*{Solution}

\begin{enumerate}[label = (\alph*)]
  \item
    A gallon makes the vehicle go 15 miles, this means that the function that describes the number of miles $ N $ that can be driven on $ x $ gallons of gas is $ N(x) = 15x $.
  \item
    TODO
  \item
    TODO
  \item
    TODO
\end{enumerate}

\subsubsection*{Answer}

\begin{enumerate}[label = (\alph*)]
  \item
    TODO
  \item
    TODO
  \item
    TODO
  \item
    TODO

\end{enumerate}
