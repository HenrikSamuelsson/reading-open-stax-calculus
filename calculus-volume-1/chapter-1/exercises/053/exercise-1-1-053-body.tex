\subsection*{Exercise 1.1.53}

\subsubsection*{Instruction}

A vehicle has a 20-gal tank and gets 15 mpg. The number of miles $ N $ that can be driven depends on the amount of gas $ x $ in the tank.

\begin{enumerate}[label = (\alph*)]
  \item
    Write a formula that models the situation.
  \item
    Determine the number of miles the vehicle can travel on (i) a full tank of gas and (ii) 3/4 of a tank of gas.
  \item
    Determine the (i) domain and (ii) range of the function.
  \item
    Determine how many times the driver had to stop for gas if she has driven a total 578 miles.
\end{enumerate}

\subsubsection*{Solution}

\begin{enumerate}[label = (\alph*)]
  \item
    A gallon makes the vehicle go 15 miles. This information leads to that the number of miles $ N $ that can be driven on $ x $ gallons of gas is $ N(x) = 15x $. This is formula is a function because it maps each input to exactly on output.
  \item
    \begin{enumerate}[label = (\roman*)]
      \item
        A full tank holds 20 gallons. This is our known $ x $, that we can use in the formula from above. The number of miles that can be traveled is $ N(20) = 15 \cdot x = 15 \cdot 20 = 300 $ miles.
      \item
        3/4 of the tank is 15 gallons. Again, this is our known $ x $, that we can use in the formula from above. The number of miles that can be traveled is $ N(15) = 15 \cdot x = 15 \cdot 15 = 225 $ miles.
    \end{enumerate}
  \item
    \begin{enumerate}[label = (\roman*)]
      \item
        The domain of the function is all the different amount of gas that is possible to put in the tank, from empty to full, $ 0 \le x \le 20 $.
      \item
        The function between the amount of gas and miles traveled is a linear relation. The more gas we have the further we can travel. With an empty tank we can travel $ N(0) = 15 \cdot x = 15 \cdot 0 = 0 $ miles. With an full tank we can travel $ N(20) = 15 \cdot x = 15 \cdot 20 = 300 $ miles. When we have something in between the extreme values in the tank we will be able to travel in between 0 and 300 miles. The range of the function is hence $ [0,300] $.
    \end{enumerate}
  \item
    We start bay calculating the number of gallons of gas required for the trip. We solve $ N = 15x $ for $ x $ by dividing both sides by 15, $ x = N/15 $. We now have a relation describing number of gallons per distance. Plug in the known distance to calculate number of gallons required, $ x = 578 / 15 \approx 39 $ gallons. If assuming that the trip was started with a full tank, holding 20 gallons, we conclude that the driver was $ 39 - 20 = 19 $ gallons short. $ 19 $ being less than one full tank means that driver had to stop at least one time during the trip to fill up the tank.
\end{enumerate}

\subsubsection*{Answer}

\begin{enumerate}[label = (\alph*)]
  \item
    $ N(x) = 15x $.
  \item
    300 miles can be traveled on a full tank of gas. 225 miles can be traveled on 3/4 of a full tank of gas.
  \item
    Domain: $ 0 \ge x \ge 20 $, range: $ [0, 300] $.
  \item
    The driver had to stop for gas refill at least once.

\end{enumerate}
