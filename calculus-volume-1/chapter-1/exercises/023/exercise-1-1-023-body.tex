\subsection*{Exercise 1.1.23}

Sketch the graph for the function $ f(x) = 3x - 6 $ with the aid of table \ref{table:exercise-1.1.23}.

\begin{table}[ht]
  \centering
  \begin{tabular}{ c | r r r r r r r }
    \hline
    $ x $ & $ -3 $ & $ -2 $ & $ -1 $ & $ 0 $ & $ 1 $ & $ 2 $ & $ 3 $ \\
    \hline
    $ y $ & $ -15 $ & $ -12 $ & $ -9 $ & $ -6 $ & $ -3 $ & $ 0 $ & $ 3 $ \\
    \hline
  \end{tabular}
  \caption{Relation between $ x $ and $ y $ in exercise 1.1.23}
  \label{table:exercise-1.1.23}
\end{table}

\subsubsection{Solution}

Begin by sketching the axes. We choose the same scale on both axes to not distort the graph. We choose the range for both axes to be -15 to 15, allowing us to plot all the points from table \ref{table:exercise-1.1.23}, see figure \ref{figure:exercise-1.1.23-figure-1}.

\begin{figure}[H]
  \centering
  \begin{tikzpicture}
    \centering
    \begin{axis}[
        axis x line = center,
        axis y line = center,
        xlabel = \( x \),
        ylabel = \( y \),
        xmin = -17.5, xmax = 17.5,
        ymin = -17.5, ymax = 17.5,
        xtick distance = 5,
        ytick distance = 5,
      ]
    \end{axis}
  \end{tikzpicture}
  \caption{Empty graph with just the axes}
  \label{figure:exercise-1.1.23-figure-1}
\end{figure}

After having sketched the axes we add markers based on the data in table \ref{table:exercise-1.1.23}, see figure \ref{figure:exercise-1.1.23-figure-2}.

\begin{figure}[H]
  \centering
  \begin{tikzpicture}
    \centering
    \begin{axis}[
        axis x line = center,
        axis y line = center,
        xlabel = \( x \),
        ylabel = \( y \),
        xmin = -17.5, xmax = 17.5,
        ymin = -17.5, ymax = 17.5,
        xtick distance = 5,
        ytick distance = 5,
      ]
      \addplot[
        only marks,
        mark = *,
      ]
      coordinates {
        (-3,-15)(-2,-12)(-1,-9)(0,-6)(1,-3)(2,0)(3,3)
      };
    \end{axis}
  \end{tikzpicture}
  \caption{Graph with added markers}
  \label{figure:exercise-1.1.23-figure-2}
\end{figure}

We then connect the markers with line segments. In this particular case the result will be a single straight line so we can use a ruler when sketching, , see figure \ref{figure:exercise-1.1.23-figure-3}.

\begin{figure}[H]
  \centering
  \begin{tikzpicture}
    \centering
    \begin{axis}[
        axis x line = center,
        axis y line = center,
        xlabel = \( x \),
        ylabel = \( y \),
        xmin = -17.5, xmax = 17.5,
        ymin = -17.5, ymax = 17.5,
        xtick distance = 5,
        ytick distance = 5,
      ]
      \addplot[
        only marks,
        mark = *,
      ]
      coordinates {
        (-3,-15)(-2,-12)(-1,-9)(0,-6)(1,-3)(2,0)(3,3)
      };
      \addplot [
        domain = -17.5:17.5,
        samples = 100,
      ]
      {3 * x - 6};
    \end{axis}
  \end{tikzpicture}
  \caption{Graph with connected markers}
  \label{figure:exercise-1.1.23-figure-3}
\end{figure}

\subsubsection*{Answer}

\begin{figure}[H]
  \centering
  \begin{tikzpicture}
    \centering
    \begin{axis}[
        axis x line = center,
        axis y line = center,
        xlabel = \( x \),
        ylabel = \( y \),
        xmin = -17.5, xmax = 17.5,
        ymin = -17.5, ymax = 17.5,
        xtick distance = 5,
        ytick distance = 5,
        legend pos = south east,
      ]
      \addplot [
        domain = -17.5:17.5,
        samples = 100,
      ]
      {3 * x - 6};
      \addlegendentry{\( 3x - 6 \)}
    \end{axis}
  \end{tikzpicture}
  \caption{Answer to exercise 1.1.23}
\end{figure}
