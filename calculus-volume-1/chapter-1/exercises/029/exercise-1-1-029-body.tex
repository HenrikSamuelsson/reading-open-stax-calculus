\subsection*{Exercise 1.1.29}

Use the vertical line test to determine whether the graph in figure \ref{figure:exercise-1.1.29} represent a function. Assume that the graph continues at both ends beyond the given grid. If the graph represents a function, then determine the following for the graph:

\begin{enumerate}[label=(\alph*)]
  \item Domain and range
  \item $ x $-intercept, if any (estimate where necessary)
  \item $ y $-intercept, if any (estimate where necessary)
  \item The intervals for which the function is increasing
  \item The intervals for which the function is decreasing
  \item The intervals for which the function is constant
  \item Symmetry about any axis and/or the origin
  \item Whether the function is even, odd, or neither
\end{enumerate}

\begin{figure}[H]
  \centering
  \begin{tikzpicture}
    \centering
    \begin{axis}[
        axis x line = center,
        axis y line = center,
        xlabel = \( x \),
        ylabel = \( y \),
        xmin = -5.5, xmax = 5.5,
        ymin = -5.5, ymax = 5.5,
        xtick distance = 1,
        ytick distance = 1,
        width = .80\textwidth
      ]
      \addplot [
        domain = -2:2,
        samples = 200,
      ]
      {x^4 - 2*x^2 + 1};
    \end{axis}
  \end{tikzpicture}
  \caption{Graph for exercise 1.1.29}
  \label{figure:exercise-1.1.29}
\end{figure}

\subsubsection{Solution}

The graph in figure \ref{figure:exercise-1.1.29} do represent a function because every vertical line that may be drawn intersects the graph no more than once. See figure \ref{figure:exercise-1.1.29-vertical-line-test} for an example of a vertical line with one intersection of the graph. We could slide this line over the entire graph and there would always only be at most one intersection.

\begin{figure}[H]
  \centering
  \begin{tikzpicture}
    \centering
    \begin{axis}[
        axis x line = center,
        axis y line = center,
        xlabel = \( x \),
        ylabel = \( y \),
        xmin = -5.5, xmax = 5.5,
        ymin = -5.5, ymax = 5.5,
        xtick distance = 1,
        ytick distance = 1,
        width = .80\textwidth
      ]
      % Plot the graph.
      \addplot [
        domain = -2:2,
        samples = 100,
      ]
      {x^4 - 2*x^2 + 1};
      % Plot a vertical line.
      \addplot[
        no marks,
      ]
      coordinates{(0.5,5.5) (0.5,-5.5)};
      % Mark the intersection between the plot and the vertical line.
      \addplot[
        mark = *,
      ]
      coordinates{(0.5,0.5625)};
    \end{axis}
  \end{tikzpicture}
  \caption{Vertical line test illustration}
  \label{figure:exercise-1.1.29-vertical-line-test}
\end{figure}

\begin{enumerate}[label=(\alph*)]
  \item
    \begin{enumerate}[label=\roman*]
      \item The function seems to grow rapidly as $ x $ goes towards $ \pm\infty $, but there will still always be a $ y $ value. We conclude that the domain is all real numbers.
      \item $ y $ is always greater or equal to $ 0 $, this is the range.
    \end{enumerate}
  \item $ y $ is zero for $ x = -1 $, and $ x = 1 $, these are the $ x $-intercepts.
  \item The $ y $-intercept is $ y = 1 $.
  \item The function is increasing for the intervals $ -1 < x < 0 $ and $ 1 < x < \infty $.
  \item The function is decreasing for the intervals $ -\infty < x < -1 $ and $ 0 < x < 1 $.
  \item The function changes from decreasing/increasing when $ x $ is $ -1 $, $ 0 $, and $ 1 $, but there are no intervals for which the function is constant.
  \item $ (-x, y) $ is on the graph whenever $ (x, y) $ is on the graph, in other words the function is symmetric around the $ y $-axis.
  \item The function is not odd because $ f(-x) \neq -f(x) $ for all $ x $ in the domain. Take for example $ x = 0.5 $ for which $ f(-x) \approx 0.6 $ and $ -f(x) \approx - 0.6 $.

    The function is even because $ f(-x) = f(x) $ for all x. Take for example  $ x = 0.5 $ for which $ f(-x) \approx 0.6 $ and $ f(x) \approx 0.6 $.
\end{enumerate}

\subsubsection*{Answer}

Graph represents a function.

\begin{enumerate}[label=(\alph*)]
  \item Domain: all real numbers, range: $ y \ge 0 $.
  \item $ x = -1 $ and $ x = 1 $.
  \item $ y = 1 $.
  \item $ -1 < x < 0 $ and $ 1 < x < \infty $.
  \item $ -\infty < x < -1 $ and $ 0 < x < 1 $.
  \item Not constant.
  \item $ y $-axis.
  \item Even.
\end{enumerate}
